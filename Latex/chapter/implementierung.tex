%!TEX root = bachelor.tex
\chapter{Implementierung}
\label{ch:implementation}

Zwecks Vereinfachung des Kalibrierungsprozesses wurde ein Assistent in Form einer graphischen Benutzerschnittstelle implementiert.

Im ersten Schritt des Assistenten wird dabei optional, wie in Abbildung \ref{fig:wizard1} zu sehen, eine intrinsische Kamerakalibrierung durchgeführt.
\begin{figure}[!htb]
	\centering
	\includegraphics[width=0.9\textwidth]{images/GUI/calibWizard1_1.PNG}
	\caption{Kalibrierungsassistent: intrinsische Kamerakalibrierung}
	\label{fig:wizard1}
\end{figure}

Anschließend wird die eigentliche Kegelkalibrierung durchgeführt. Dabei wird zunächst das Kalibrierungsbild geladen und gegebenenfalls nach einer stattgefundenen intrinsischen Kalibrierung entzerrt. Die Anwendung versucht nun die Sample-Positionen zu erkennen und der Nutzer hat die Möglichkeit zu überprüfen, ob alle Positionen korrekt detektiert wurden und andernfalls fehlerhafte Punkte zu entfernen und / oder Samplepositionen hinzuzufügen. Im Anschluss werden die Ellipsen und Liniensegmente bestimmt und Punktkorrespondenzen hergestellt (siehe Abbildung \ref{fig:wizard2}).

\begin{figure}[!htb]
	\centering
	\includegraphics[width=0.9\textwidth]{images/GUI/calibWizard2_1.PNG}
	\caption{Kalibrierungsassistent: Kegelkalibrierung}
	\label{fig:wizard2}
\end{figure}

Im letzten Schritt kann zwischen beiden Entfaltungsverfahren gewählt werden. Anschließend können die Einstellungen in eine XML-Datei exportiert werden, in der, neben der Kamera-Matrix und den Verzerrungskoeffizienten, auch die zwei Abbildungsmatrizen des ausgewählten Verfahrens gespeichert werden.


Die Abbildungsmatrizen sind dabei wie folgt aufgebaut: als $dst$ wird das entfaltete Bild bezeichnet, $src$ entspricht dem Ursprungsbild.

Bei der Vowärtsentfaltung gilt:
\[
dst(map_x(x,y), map_y(x,y)) = src(x,y),
\]

wobei $map_x$ und $map_y$ die Abbildungsmatrizen sind und die gleiche Größe wie das Ursprungsbild haben. Für ein Pixel $(x,y)$ auf dem Ursprungsbild $src$ geben die Matrizen $map_x$, sowie $map_y$ an, auf welche Position das Pixel auf dem entzerrten Bild abgebildet wird. $map_x(x,y)$ bestimmt dabei die $x$-Koordinate, $map_y(x,y)$ die $y$-Koordinate. Da bei der Entfaltung die Größe des Ergebnisbildes benötigt wird, und diese bei der Erstellung der Abbildungsmatrizen berechnet wurde, sind Breite und Höhe, in $map_x(0,0)$, respektive $map_y(0,0)$ kodiert. Für das Pixel an der Position $(0,0)$ auf dem Ursprungsbild ist anschließend keine Auswertung der Matrizen mehr möglich. Dieses Pixel ist jedoch ohnehin nicht Teil des Kegels.
\newpage
Bei der Rückwärtsentfaltung gilt:
\[
dst(x,y) = src((map_x(x,y),map_y(x,y)),
\]
wobei $map_x$ und $map_y$ wieder die Abbildungsmatrizen sind, deren Aufbau analog zur Vorwärtsenfaltung ist. 
Der grundlegende Unterschied ist hier, dass von einem Punkt des entzerrten Bilds auf ein Pixel des Ursprungsbild geschlossen wird. 
Die Abbildungsmatrizen haben die gleiche Größe wie das entfaltete Bild. Eine Kodierung wie bei der Vowärtsentfaltung ist nicht notwendig.
