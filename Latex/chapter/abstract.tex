\begin{abstract}
\section*{Zusammenfassung}
Zur Bewegungsanalyse von Kriechtieren insbesondere der Larven der Drosphila melanogaster,
 wurde ein Untersuchungsaufbau konstruiert, so dass diese in einem aus Glas gefertigten Kegelstumpf
 mit einer Weitwinkelkamera untersucht werden können. Der Kegelstumpf ist dabei mit der Grundfläche
 nach oben ausgerichtet. Die Kamera nimmt aus der Vogelperspektive auf.
Auf Grund der perspektivischen Verzerrung scheinen jedoch Versuchsobjekte im unteren Teil des Kegelstumpfes kleiner,
als jene die sich weiter oben befinden. Ein relativer Vergleich der Objekte ist nicht möglich.

Ziel dieser Arbeit ist die Entwicklung eines Verfahrens zur Entzerrung
solcher Kegeloberflächen, so dass die relative Vergleichbarkeit der Versuchsobjekte gewährleistetet ist.
Mittels RANSAC zur robusten Ellipsenschätzung und Hough-Transformation zur Liniendetektion
werden dabei anhand eines Kalibrierungsmusters Korrespondenzen zwischen den Bildpunkten,
sowie den Punkte des Kegels im Weltkoordinatensystem hergestellt. Anschließend wird das Bild des Kegelstumpfs
mit Hilfe einer geeigneten Abbildung auf dessen Mantelfläche abbgebildet.
Das Ergebnis ist ein robustes Verfahren, dass auch bei leichten Abweichungen der Kamera vom Lot gute Ergebnisse
liefert.
\end{abstract}
